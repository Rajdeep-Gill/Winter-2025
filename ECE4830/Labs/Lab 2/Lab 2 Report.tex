\documentclass{article}
\usepackage[margin=2.5cm, top=4cm, headheight=25pt]{geometry}
\usepackage{amsmath, amssymb, enumitem, fancyhdr, graphicx}
\usepackage[indent=20pt]{parskip}
\usepackage[hidelinks]{hyperref}
\usepackage{xcolor}
\usepackage{listings}
\usepackage{subcaption}
\usepackage{url}
\usepackage[most]{tcolorbox}
\usepackage{lastpage}

\tcbuselibrary{listingsutf8} % Support for lstlistings within tcolorbox

\newtcolorbox[auto counter, number within=section]{question}[1][]{%
    colframe=gray!80,                      % Dark gray frame
    colback=gray!5,                       % Light gray background
    coltitle=black,                        % Black title
    title=\textbf{Question~\thetcbcounter}, % Bold title
    fonttitle=\bfseries\large,             % Subtle title font size
    rounded corners,                   % Slightly more rounded corners
    boxrule=0.25mm,                         % Thinner border for a sleek look
    enhanced,                              % Enhanced box features
    attach boxed title to top left={xshift=2mm, yshift=-2mm},
    boxed title style={colframe=gray!80, colback=gray!5, boxrule=0.25mm},
    % Title styling
    #1
}

\bibliographystyle{IEEEtran}
\graphicspath{{./images/}}

% -- Custom Variables --
\def\me{Rajdeep Gill (7934493)}
\def\partner{Daniyal Hasnain (7942244)}
\def\course{ECE 4830}
\def\labsection{B03}
\def\labno{2}
\def\title{Lab 2}

% -- Styling for code snippets --
\lstset{
    basicstyle=\ttfamily\small,           % Basic font style
    keywordstyle=\color{blue},            % Keywords color
    commentstyle=\color{gray},            % Comments color
    stringstyle=\color{teal},             % Strings color
    numbers=left,                         % Line numbers on the left
    numberstyle=\tiny\color{gray},        % Line number style
    stepnumber=1,                         % Line number step
    numbersep=10pt,                       % Space between line numbers and code
    backgroundcolor=\color{lightgray!10}, % Background color
    frame=single,                         % Adds a frame around the code
    breaklines=true,                      % Line breaking for long lines
    captionpos=b,                         % Caption position
    showspaces=false,                     % Don't show spaces
    showstringspaces=false                % Don't show spaces in strings
}
\renewcommand{\lstlistingname}{Code Snippet}

\renewcommand{\arraystretch}{1.2} % For less-ugly tables
\setlength\parindent{0pt}

%----- Samples 
% Questions:
%   \begin{question}[title=Custom Question Title]
%       Question details
%   \end{question}

% Tables:
%   \begin{table}[htbp]
%       \centering
%       \caption{Table Caption}
%       \begin{tabular}{ll}
%           \toprule
%           \textbf{Column 1} & \textbf{Column 2} \\
%           \midrule
%           Row 1 & Row 2 \\
%           Row 3 & Row 4 \\
%           \bottomrule
%       \end{tabular}
%   \end{table} 

% Figures:
%   Single figure:
%       \begin{figure}[htbp]
%           \centering
%           \includegraphics[width=0.5\textwidth]{example-image}
%           \caption{Figure Caption}
%       \end{figure}
%   Multiple figures:
%       \begin{figure}[htbp]
%           \centering
%           \begin{subfigure}[b]{0.5\textwidth}
%               \includegraphics[width=\textwidth]{example-image-a}
%               \caption{First subfigure}
%           \end{subfigure}
%           \begin{subfigure}[b]{0.5\textwidth}
%               \includegraphics[width=\textwidth]{example-image-b}
%               \caption{Second subfigure}
%           \end{subfigure}
%           \caption{Main figure}
%       \end{figure}

\begin{document}

% --------------------------------------------------------------------------------
% TITLE
% --------------------------------------------------------------------------------

\begin{center}
    \huge \title

    \vspace{2mm}
    \hrule

    \vspace{4mm}
    \large \me ~\&~\partner

    \vspace{2mm}
    \large \course~\labsection

    \vspace{2mm}
    \today
\end{center}

\vspace{4mm}

% --------------------------------------------------------------------------------
% END TITLE
% --------------------------------------------------------------------------------

\newpage

\tableofcontents

\vspace{1cm}
\newpage

\pagestyle{fancy}
\fancyhead[L]{\large Lab \labno}
\fancyhead[R]{\large \me, \partner}

\fancyfoot[C]{Page \thepage~of~\pageref{LastPage}}

% --------------------------------------------------------------------------------
% BODY
% --------------------------------------------------------------------------------

\section{Problem 1}

Given system, we can find the individual impulse responses and then sum them to find the impulse response of the entire system. We only wish to find the first three samples in time, that is $h(0)$, $h(1)$, and $h(2)$.

We have that $y_1(n) - \frac{1}{2}y_1(n-1) = 2x(n)$, and $h_2(n) = \frac{1}{4^n}u(n)$.

The impulse response of the first system can be found by the following:

\begin{align*}
    h_1(n) - \frac{1}{2}h_1(n-1) &= 2\delta(n) \\
    h(n) &= A_0\delta(n) + y_c(n)u(n)
\end{align*}
We know that $A_0 = \frac{0}{1/2} = 0$, and $y_c(n)$ is the characteristic equation. That is:
\begin{align*}
    \lambda - \frac{1}{2} &= 0 \\
    \implies \lambda &= \frac{1}{2} \\
    y_c(n) &= A\left(\frac{1}{2}\right)^n
\end{align*}

We can find $A$ by first finding $h_1(0)$:
\begin{align*}
    n=0: h_1(0) - \frac{1}{2}h_1(-1) &= 2\delta(0) \\
    h_1(0) &= 2 \\
\end{align*}

Plugging this in, we get $A$, and have our characteristic equation:
\begin{align*}
    h_1(0) &= A\left(\frac{1}{2}\right)^0 = 2 \\
    A &= 2 \\
    y_c(n) &= 2\left(\frac{1}{2}\right)^nu(n)
\end{align*}

We are already given the impulse response of the second system, so we can sum the two to get the impulse response of the entire system:
\begin{align*}
    h(n) &= h_1(n) + h_2(n) \\ 
    &= \left[2\left(\frac{1}{2}\right)^n + \frac{1}{4^n}\right]u(n)
\end{align*}

The first three samples are then:
\begin{align*}
    h(0) &= 2\left(\frac{1}{2}\right)^0 + \frac{1}{4^0} = 2 + 1 = 3 \\
    h(1) &= 2\left(\frac{1}{2}\right)^1 + \frac{1}{4^1} = 1 + 0.25 = 1.25 \\
    h(2) &= 2\left(\frac{1}{2}\right)^2 + \frac{1}{4^2} = 0.5 + 0.0625 = 0.5625
\end{align*}


We can double check our solution with the following Matlab code:
\begin{lstlisting}[language=Matlab]
    x = [1, zeros(1, 9)];
    
    % First system
    b1 = [2];
    a1 = [1, -0.5];
    h1 = filter(b1, a1, x);
    
    % Second system
    n = 0:N-1;
    h2 = (1./(4.^n));
    
    % Total impulse response
    h = h1 + h2;
    
    fprintf('h(0) = %.4f\n', h(1));
    fprintf('h(1) = %.4f\n', h(2));
    fprintf('h(2) = %.4f\n', h(3));
\end{lstlisting}

We get the following output, and can see that it is the same as we calculated:
\begin{verbatim}
    h(0) = 3.0000
    h(1) = 1.2500
    h(2) = 0.5625
\end{verbatim}

\section{Problem 2}
\begin{enumerate}[label=(\alph*)]
    \item If initial conditions are zero, the zero-input response would yield the impulse response
    
    \textbf{No}, because the zero-input response means, by definition, that the input is zero. The impulse response is the response of the system to $\delta(n)$, which is not zero.

    \item If initial conditions are non-zero, then I can use these initial condition values to compute the impulse response 
    
    \textbf{No}, because the impulse response is the response of the system to $\delta(n)$, and the initial conditions are not used to compute the impulse response.

    \item I can compute the output of a system $y(n)$ for an input $x(n)$ using a convolution if I know the impulse response of the system
    
    \textbf{Yes}, as long as the system is an LTI system and the input is known, the output can be computed using convolution.

    \item I can compute the impulse response of a system using $h(n) = \frac{b_M}{a_N}\delta(n) + y_c(n)u(n)$ if the order associated to the input is greater than the output, that is $M > N$

    \textbf{No}, because if $M > N$, then we would have $a_N = 0$, and we would be dividing by zero. We require $M \leq N$.
\end{enumerate}

\section{Problem 3}

\begin{enumerate}[label=(\alph*)]
    \item Using the width property, we kow that the convolution, $c(n)$, will range from $4+(-2) = 2$ to $N+2$. From the following options, a to i, we can see that $c(n)$ is non zero for $n \geq 2$ and $n \leq N+2$. The option that satisfies this is \textbf{g}.
    \item The final length of the convolution is $N+2 - 2 + 1 = N+1$. The option that satisfies this is \textbf{i}.
\end{enumerate}

\section{Problem 4}
Find the expression of the zero-impulse response of the the system:
\[
    y(n) + \frac{1}{2}y(n-1)-\frac{3}{4}y(n-2) = -\frac{3}{2}x(n) + \frac{3}{4}x(n-2)
\]
\[
    y(-1) = 2, \quad y(-2) = 4
\]
We were unsure what the zero-impulse response was, so after asking the TA, and fellow students, we were told it was a typo and should be zero-input response. We will solve for the zero-input response instead.
\begin{align*}
    y(n) + \frac{1}{2}y(n-1)-\frac{3}{4}y(n-2) &= 0 \\
    \text{Characteristic equation: } \lambda^2 + \frac{1}{2}\lambda - \frac{3}{4} &= 0 \\
    \lambda &= \frac{-1/2 \pm \sqrt{13/4}}{2} = \frac{-1}{4} \pm \frac{\sqrt{13}}{4} \\
    \implies y_c(n) &= C_1\left(\frac{-1}{4} + \frac{\sqrt{13}}{4}\right)^n + C_2\left(\frac{-1}{4} - \frac{\sqrt{13}}{4}\right)^n \\
    &= C_1\left(0.6514\right)^n + C_2\left(-1.1514\right)^n
\end{align*}

Solving for $A$ and $B$:
\begin{align*}
    y(-1) = 2 &= C_1\left(0.6514\right)^{-1} + C_2\left(-1.1514\right)^{-1}\\
    2&= 1.5352C_1 - 0.8685C_2 \\
    y(-2) = 4 &=  C_1\left(0.6514\right)^{-2} + C_2\left(-1.1514\right)^{-2}\\
    4&= 2.3568C_1 + 0.7543C_2
\end{align*}

We have the following system of equations:
\begin{align*}
    \begin{bmatrix}
        1.5352 & -0.8685 \\
        2.3568 & 0.7543
    \end{bmatrix} \begin{bmatrix}
        A \\ B
    \end{bmatrix} &= \begin{bmatrix}
        2 \\ 4
    \end{bmatrix} \\
    \textbf{A}\textbf{x} &= \textbf{b} \\
\end{align*}
Using Cramer's rule, we have:
\begin{align*}
    \text{det}(\textbf{A}) &= (1.5353 \times 0.7543) - (-0.8685 \times 2.3568) = 3.2049 \\
    \text{det}(\textbf{A}_{C_1}) &= \left|
        \begin{matrix}
            2 & -0.8685 \\
            4 & 0.7543
        \end{matrix}
    \right| = (2\times0.7543) - (4 \times -0.8682) = 4.9814 \\
    \text{det}(\textbf{A}_{C_2}) &= \left|
        \begin{matrix}
            1.5352 & 2 \\
            2.3568 & 4
        \end{matrix}
    \right| = (1.5352 \times 4) - (2.3568 \times 2) = 1.4272 \\
    C_1 &= \frac{\text{det}(\textbf{A}_{C_1})}{\text{det}(\textbf{A})} = \frac{4.9814}{3.2049} = 1.5536 \\
    C_2 &= \frac{\text{det}(\textbf{A}_{C_2})}{\text{det}(\textbf{A})} = \frac{1.4272}{3.2049} = 0.4453
\end{align*}

The zero-input response is then:
\begin{align*}
    y_{zi}(n) &= 1.5536\left(0.6514\right)^n + 0.4453\left(-1.1514\right)^n
\end{align*}

Using the following Matlab code, we can verify closed form zero input response to the reccurence relation:
\begin{lstlisting}[language=MATLAB]
n_vals = -2:10; 

% Initialize y 
y = zeros(1, length(n_vals));  
y(1) = 4;  % y(-2)
y(2) = 2;  % y(-1)
% Compute y(n) using difference equation
for n = 3:length(n_vals)
    y(n) = -0.5 * y(n-1) + 0.75 * y(n-2);
end

% Compute analytical zero-input response
y_zi = 1.5536 * (0.6514.^(n_vals)) + 0.4453 * ((-1.1514).^(n_vals));

disp(' n    Recurrence  Analytical  Difference');
disp([n_vals' y' y_zi' abs(y - y_zi)']);
\end{lstlisting}

We get the following output, with a very small difference between the two results at each index. This difference is due to floating point errors and rounded values.
\begin{verbatim}
    n       Recurrence  Analytical  Difference
    -2      4.0000      3.9973      0.0027
    -1      2.0000      1.9983      0.0017
    0       2.0000      1.9989      0.0011
    1       0.5000      0.4993      0.0007
    2       1.2500      1.2496      0.0004
    3       -0.250      -0.250      0.0003
    4       1.0625      1.0624      0.0001
    5       -0.719      -0.719      0.0002
    6       1.1562      1.1562      0.0000
    7       -1.117      -1.117      0.0001
    8       1.4258      1.4259      0.0001
    9       -1.551      -1.551      0.0002
   10       1.8447      1.8449      0.0002
\end{verbatim}

% --------------------------------------------------------------------------------
% END BODY
% --------------------------------------------------------------------------------

\end{document}
