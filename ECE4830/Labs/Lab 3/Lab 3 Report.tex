\documentclass{article}
\usepackage[margin=2.5cm, top=4cm, headheight=25pt]{geometry}
\usepackage{amsmath, amssymb, enumitem, fancyhdr, graphicx}
\usepackage[indent=20pt]{parskip}
\usepackage[hidelinks]{hyperref}
\usepackage{xcolor}
\usepackage{listings}
\usepackage{subcaption}
\usepackage{url}
\usepackage[most]{tcolorbox}
\usepackage{lastpage}

\tcbuselibrary{listingsutf8} % Support for lstlistings within tcolorbox

\newtcolorbox[auto counter, number within=section]{question}[1][]{%
    colframe=gray!80,                      % Dark gray frame
    colback=gray!5,                       % Light gray background
    coltitle=black,                        % Black title
    title=\textbf{Question~\thetcbcounter}, % Bold title
    fonttitle=\bfseries\large,             % Subtle title font size
    rounded corners,                   % Slightly more rounded corners
    boxrule=0.25mm,                         % Thinner border for a sleek look
    enhanced,                              % Enhanced box features
    attach boxed title to top left={xshift=2mm, yshift=-2mm},
    boxed title style={colframe=gray!80, colback=gray!5, boxrule=0.25mm},
    % Title styling
    #1
}

\bibliographystyle{IEEEtran}
\graphicspath{{./images/}}

% -- Custom Variables --
\def\me{Rajdeep Gill (7934493)}
\def\partner{Daniyal Hasnain (7942244)}
\def\course{ECE 4830}
\def\labsection{B03}
\def\labno{2}
\def\title{Lab 2}

% -- Styling for code snippets --
\lstset{
    basicstyle=\ttfamily\small,           % Basic font style
    keywordstyle=\color{blue},            % Keywords color
    commentstyle=\color{gray},            % Comments color
    stringstyle=\color{teal},             % Strings color
    numbers=left,                         % Line numbers on the left
    numberstyle=\tiny\color{gray},        % Line number style
    stepnumber=1,                         % Line number step
    numbersep=10pt,                       % Space between line numbers and code
    backgroundcolor=\color{lightgray!10}, % Background color
    frame=single,                         % Adds a frame around the code
    breaklines=true,                      % Line breaking for long lines
    captionpos=b,                         % Caption position
    showspaces=false,                     % Don't show spaces
    showstringspaces=false                % Don't show spaces in strings
}
\renewcommand{\lstlistingname}{Code Snippet}

\renewcommand{\arraystretch}{1.2} % For less-ugly tables
\setlength\parindent{0pt}

%----- Samples 
% Questions:
%   \begin{question}[title=Custom Question Title]
%       Question details
%   \end{question}

% Tables:
%   \begin{table}[htbp]
%       \centering
%       \caption{Table Caption}
%       \begin{tabular}{ll}
%           \toprule
%           \textbf{Column 1} & \textbf{Column 2} \\
%           \midrule
%           Row 1 & Row 2 \\
%           Row 3 & Row 4 \\
%           \bottomrule
%       \end{tabular}
%   \end{table} 

% Figures:
%   Single figure:
%       \begin{figure}[htbp]
%           \centering
%           \includegraphics[width=0.5\textwidth]{example-image}
%           \caption{Figure Caption}
%       \end{figure}
%   Multiple figures:
%       \begin{figure}[htbp]
%           \centering
%           \begin{subfigure}[b]{0.5\textwidth}
%               \includegraphics[width=\textwidth]{example-image-a}
%               \caption{First subfigure}
%           \end{subfigure}
%           \begin{subfigure}[b]{0.5\textwidth}
%               \includegraphics[width=\textwidth]{example-image-b}
%               \caption{Second subfigure}
%           \end{subfigure}
%           \caption{Main figure}
%       \end{figure}

\begin{document}

% --------------------------------------------------------------------------------
% TITLE
% --------------------------------------------------------------------------------

\begin{center}
    \huge \title

    \vspace{2mm}
    \hrule

    \vspace{4mm}
    \large \me ~\&~\partner

    \vspace{2mm}
    \large \course~\labsection

    \vspace{2mm}
    \today
\end{center}

\vspace{4mm}

% --------------------------------------------------------------------------------
% END TITLE
% --------------------------------------------------------------------------------

\newpage

\tableofcontents

\vspace{1cm}
\newpage

\pagestyle{fancy}
\fancyhead[L]{\large Lab \labno}
\fancyhead[R]{\large \me, \partner}

\fancyfoot[C]{Page \thepage~of~\pageref{LastPage}}

% --------------------------------------------------------------------------------
% BODY
% --------------------------------------------------------------------------------

\section{Problem 1}

\section{Problem 2}

\section{Problem 3}

The given fourier transform is a triangular function defined as follows:
\begin{align*}
    X(\omega) &= \begin{cases}
        1 + \frac{4}{\pi} \omega & \text{if } -\frac{\pi}{4} \leq \omega \leq 0 \\
        1 - \frac{4}{\pi} \omega & \text{if } 0 \leq \omega \leq \frac{\pi}{4} \\
    \end{cases}
\end{align*}

Using the differentiation property:
\begin{align*}
    nx(n) \leftrightarrow j \frac{dX(\omega)}{d\omega}
\end{align*}

The derivative of the given triangular function is:
\begin{align*}
    \frac{dX(\omega)}{d\omega} &= \begin{cases}
        \frac{4}{\pi} & \text{if } -\frac{\pi}{4} \leq \omega \leq 0 \\
        -\frac{4}{\pi} & \text{if } 0 \leq \omega \leq \frac{\pi}{4} \\
    \end{cases}
\end{align*}

The inverse fourier transform of the derivative is:
\begin{align*}
    nx(n) &= j \frac{1}{2\pi} \int_{-\pi}^{\pi} \frac{dX(\omega)}{d\omega} e^{j \omega n} d\omega \\
    &= j \frac{1}{2\pi}\int_{-\frac{\pi}{4}}^{0} \frac{4}{\pi} e^{j \omega n} d\omega + j \int_{0}^{\frac{\pi}{4}} -\frac{4}{\pi} e^{j \omega n} d\omega \\
    &= j \frac{1}{2\pi}\left[ \frac{4}{\pi jn} e^{j \omega n} \right]_{-\frac{\pi}{4}}^{0} - j \left[ \frac{4}{\pi jn} e^{j \omega n} \right]_{0}^{\frac{\pi}{4}} \\
    &= \frac{2}{\pi^2 n} \left[ \left(e^{j\omega n}\right)\Bigg|_{-\frac{\pi}{4}}^{0} - \left(e^{j\omega n}\right)\Bigg|_{0}^{\frac{\pi}{4}} \right] \\
    nx(n) &= \frac{2}{\pi^2 n} \left[ 1 - e^{-j\frac{\pi}{4} n} - e^{j\frac{\pi}{4} n} + 1 \right] \\
    x(n)&= \frac{2}{\pi^2 n^2} \left[ 2 - e^{-j\frac{\pi}{4} n} - e^{j\frac{\pi}{4} n} \right] \\
    &= \frac{2}{\pi^2 n^2} \left[2 - 2\cos\left(\frac{\pi}{4}n\right)\right] \\
    &= \frac{8}{\pi^2 n^2} \sin^2\left(\frac{\pi}{8}n\right) \\
    &= \frac{8}{\pi^2} \times \frac{\sin\left(\frac{\pi}{8}n\right)}{n} \times \frac{\sin\left(\frac{\pi}{8}n\right)}{n} \\
    &= \frac{8}{\pi^2}  \times \frac{\frac{\pi}{8} \sin\left(\frac{\pi}{8}n\right)}{\frac{\pi}{8}n} \times \frac{\frac{\pi}{8} \sin\left(\frac{\pi}{8}n\right)}{\frac{\pi}{8}n} \\
    &= \frac{8}{\pi^2}  \times \frac{\pi}{8} \text{sinc}\left(\frac{n}{8}\right) \times \frac{\pi}{8} \text{sinc}\left(\frac{n}{8}\right) \\
    &= \boxed{\frac{1}{8}\text{sinc}\left(\frac{n}{8}\right)^2}
\end{align*}


\section{Problem 4}

The convolution of the two provided signals, can be easily found by first taking the fourier transform of each signal, multiplying them together, and then taking the inverse fourier transform of the result.
\begin{align*}
    x_1(n) = \frac{A}{16} \text{sinc} \left(\frac{\pi}{16}n\right), \quad x_2(n) = B\cos\left(\frac{3\pi}{4}n\right)
\end{align*}

The fourier transform of a sinc function is a rectangular function, and the fourier transform of a cosine function is a pair of impulses. The fourier transform of the first signal can be found by using the following properties:
\begin{align*}
    \text{sinc}(n) &\xrightarrow{\mathcal{F}} \text{rect}(f) \\
    x(an) &\xrightarrow{\mathcal{F}} \frac{1}{|a|}X\left(\frac{f}{a}\right)
\end{align*}

Applying these properties to the first signal:
\begin{align*}
    \frac{A}{16}\text{sinc}\left(\frac{\pi}{16}n\right) &\xrightarrow{\mathcal{F}} \frac{A}{\pi}\text{rect}\left(\frac{16}{\pi} f\right) \\
\end{align*}

The fourier transform of the second signal is a pair of impulses at $f = \pm \frac{3}{8}$.
\begin{align*}
    B\cos\left(\frac{3\pi}{4}n\right) &\xrightarrow{\mathcal{F}} \frac{B}{2}\left[\delta\left(f + \frac{3}{8}\right) + \delta\left(f - \frac{3}{8}\right)\right]
\end{align*}

The fourier transform of the convolution of the two signals is the product of the fourier transforms of the two signals.
\begin{align*}
    Y(f) &= \frac{A}{\pi}\text{rect}\left(\frac{16}{\pi} f\right) \times \frac{B}{2}\left[\delta\left(f + \frac{3}{8}\right) + \delta\left(f - \frac{3}{8}\right)\right] \\
    &= \frac{AB}{2\pi}\text{rect}\left(\frac{16}{\pi} f\right) \left[\delta\left(f + \frac{3}{8}\right) + \delta\left(f - \frac{3}{8}\right)\right]
\end{align*}

The inverse fourier transform is then:
\begin{align*}
    y(n) &= \mathcal{F}^{-1}\left\{Y(f)\right\} \\
    &= \frac{AB}{2\pi} \int_{-\infty}^{\infty} \text{rect}\left(\frac{16}{\pi} f\right) \left[\delta\left(f + \frac{3}{8}\right) + \delta\left(f - \frac{3}{8}\right)\right] e^{j2\pi fn} df \\
\end{align*}

The rectangular function will be 1 when $ \left|\frac{16}{\pi} f\right| \leq \frac{1}{2}$, and $0$ otherwise. This means that our integral will only be non-zero when $-\frac{\pi}{32} \leq f \leq \frac{\pi}{32}$. 
\begin{align*}
    y(n) &= \frac{AB}{2\pi} \int_{-\frac{\pi}{32}}^{\frac{\pi}{32}} \left[\delta\left(f + \frac{3}{8}\right) + \delta\left(f - \frac{3}{8}\right)\right] e^{j2\pi fn} df \\
\end{align*}

Since $\frac{3}{8}$ and $-\frac{3}{8}$ are outside of the range of integration, the integral will evaluate to 0. This means that the convolution of the two signals is 0.

\section{Problem 5}

Given the output $y(n)$ in the frequency domain:
\begin{align*}
    Y(\omega) = e^{-j2\pi \omega}X(\omega) + \frac{dX(\omega)}{d\omega}
\end{align*}

\begin{enumerate}[label=(\alph*)]
    \item \textit{Compute the response of the system to the input $x(n) = \delta(n)$.}

    The fourier transform of the impulse function is 1. Substituting this into the given equation:
    \begin{align*}
        Y(\omega) &= e^{-j2\pi \omega}X(\omega) + \frac{dX(\omega)}{d\omega} \\
        Y(\omega) &= e^{-j2\pi \omega} + \frac{d}{d\omega}1 \\
        Y(\omega) &= e^{-j2\pi \omega} \\
    \end{align*}
    The inverse fourier transform of $e^{-j2\pi \omega}$ is $\delta(n-2\pi)$. This means that the response of the system to the input $x(n) = \delta(n)$ is $\boxed{\delta(n-2\pi)}$.
    
    However, since the shift is by $2\pi$, and we are in the discrete domain, the response is $y(n) = 0$.


    \item \textit{What is the difference equation in time of the system?}

    The difference equation can be found by taking the inverse fourier transform of the given equation. 

    We will use the derivative property and the time shift property of the fourier transform:
    \begin{align*}
        x(t-t_0) &\xrightarrow{\mathcal{F}} e^{-j\omega t_0}X(\omega) \\
        nx(t) &\xrightarrow{\mathcal{F}} j \frac{dX(\omega)}{d\omega} \implies \frac{dX(\omega)}{d\omega} \xrightarrow{\mathcal{F}^{-1}} jnx(t)
    \end{align*}

    Using these properties, we can find the inverse fourier transform of $Y(\omega)$:
    \begin{align*}
        y(n) &= \mathcal{F}^{-1}\left\{Y(\omega)\right\} \\
        y(n) &= \mathcal{F}^{-1}\left\{e^{-j2\pi \omega}X(\omega)\right\} + \mathcal{F}^{-1}\left\{\frac{dX(\omega)}{d\omega}\right\} \\
        y(n) &= x(n-1) + jnx(n) \\
    \end{align*}

    \item \textit{Check if the system is stable and time-invariant.}

    To check for time-invariance, we can check if the system response to a time-shifted input is the same as the time-shifted response to the original input.
    \begin{align*}
        x_1(n) &= \delta(n-1) \implies y_1(n) = \delta(n-2) + jn\delta(n-1) \\
        y(n-1) &= \delta(n-2) + j(n-1)\delta(n-2) \\
        y(n-1) &\neq y_1(n) \implies \text{System is not time-invariant}
    \end{align*}

    Due to the presence of the $n$ term in the second equation, the system is not time-invariant.

    We see that the impulse response of the system is $\delta(n-2\pi)$ from part (a). This means that the system is stable as:
    \begin{align*}
        \sum_{n=-\infty}^{\infty} |\delta(n-2\pi)| = 0 < \infty
    \end{align*}
\end{enumerate}


We are given:
\begin{align*}
    Y(\omega) = e^{-j2\pi \omega}X(\omega) + \frac{dX(\omega)}{d\omega}
\end{align*}

Would it be better to interpret it as frequency instead of angular frequency?
\begin{align*}
    Y(f) = e^{-j2\pi f}X(f) + \frac{dX(f)}{df}
\end{align*}


% --------------------------------------------------------------------------------
% END BODY
% --------------------------------------------------------------------------------

\end{document}
