\documentclass{article}
\usepackage[margin=2.5cm, top=4cm, headheight=25pt]{geometry}
\usepackage{amsmath, amssymb, enumitem, fancyhdr, graphicx}
\usepackage[indent=20pt]{parskip}
\usepackage[hidelinks]{hyperref}
\usepackage{xcolor}
\usepackage{listings}
\usepackage{subcaption}
\usepackage{url}
\usepackage[most]{tcolorbox}
\usepackage{lastpage}

\tcbuselibrary{listingsutf8} % Support for lstlistings within tcolorbox

\newtcolorbox[auto counter, number within=section]{question}[1][]{%
    colframe=gray!80,                      % Dark gray frame
    colback=gray!5,                       % Light gray background
    coltitle=black,                        % Black title
    title=\textbf{Question~\thetcbcounter}, % Bold title
    fonttitle=\bfseries\large,             % Subtle title font size
    rounded corners,                   % Slightly more rounded corners
    boxrule=0.25mm,                         % Thinner border for a sleek look
    enhanced,                              % Enhanced box features
    attach boxed title to top left={xshift=2mm, yshift=-2mm},
    boxed title style={colframe=gray!80, colback=gray!5, boxrule=0.25mm},
    % Title styling
    #1
}

\bibliographystyle{IEEEtran}
\graphicspath{{./images/}}

% -- Custom Variables --
\def\me{Rajdeep Gill 7934493}
\def\course{ECE 4260}
\def\labsection{A01}

\def\labno{3}
\def\title{Assignment 3}

% -- Styling for code snippets --
\lstset{
    basicstyle=\ttfamily\scriptsize,           % Basic font style
    keywordstyle=\color{blue},            % Keywords color
    commentstyle=\color{gray},            % Comments color
    stringstyle=\color{teal},             % Strings color
    numbers=left,                         % Line numbers on the left
    numberstyle=\tiny\color{gray},        % Line number style
    stepnumber=1,                         % Line number step
    numbersep=10pt,                       % Space between line numbers and code
    backgroundcolor=\color{lightgray!10}, % Background color
    frame=single,                         % Adds a frame around the code
    breaklines=true,                      % Line breaking for long lines
    captionpos=b,                         % Caption position
    showspaces=false,                     % Don't show spaces
    showstringspaces=false                % Don't show spaces in strings
}
\renewcommand{\lstlistingname}{Code Snippet}

\renewcommand{\arraystretch}{1.2} % For less-ugly tables
\setlength\parindent{0pt}

%----- Samples 
% Questions:
%   \begin{question}[title=Custom Question Title]
%       Question details
%   \end{question}

% Tables:
%   \begin{table}[htbp]
%       \centering
%       \caption{Table Caption}
%       \begin{tabular}{ll}
%           \toprule
%           \textbf{Column 1} & \textbf{Column 2} \\
%           \midrule
%           Row 1 & Row 2 \\
%           Row 3 & Row 4 \\
%           \bottomrule
%       \end{tabular}
%   \end{table} 

% Figures:
%   Single figure:
%       \begin{figure}[htbp]
%           \centering
%           \includegraphics[width=0.5\textwidth]{example-image}
%           \caption{Figure Caption}
%       \end{figure}
%   Multiple figures:
%       \begin{figure}[htbp]
%           \centering
%           \begin{subfigure}[b]{0.5\textwidth}
%               \includegraphics[width=\textwidth]{example-image-a}
%               \caption{First subfigure}
%           \end{subfigure}
%           \begin{subfigure}[b]{0.5\textwidth}
%               \includegraphics[width=\textwidth]{example-image-b}
%               \caption{Second subfigure}
%           \end{subfigure}
%           \caption{Main figure}
%       \end{figure}

\begin{document}

% --------------------------------------------------------------------------------
% TITLE
% --------------------------------------------------------------------------------

\begin{center}
    \huge \title

    \vspace{2mm}
    \hrule

    \vspace{4mm}
    \large \me

    \vspace{2mm}
    \large \course~\labsection

    \vspace{2mm}
    \today
\end{center}

\vspace{4mm}

% --------------------------------------------------------------------------------
% END TITLE
% --------------------------------------------------------------------------------

\newpage


\vspace{1cm}
\newpage

\pagestyle{fancy}
\fancyhead[L]{\large Assignment \labno}
\fancyhead[R]{\large \me}

\fancyfoot[C]{Page \thepage~of~\pageref{LastPage}}

% --------------------------------------------------------------------------------
% BODY
% --------------------------------------------------------------------------------
\section{Problem 1}

\begin{enumerate}[label=1.\arabic*]
    \item We can show the following:
    \begin{align*}
        \int_{-\infty}^{\infty} g_1(t)g_2^*(t) \, dt &= \int_{-\infty}^{\infty} G_1(f)G_2^*(f) \, df
    \end{align*}

    Starting with the left-hand side:
    \begin{align*}
        \int_{-\infty}^{\infty} g_1(t)g_2^*(t) \, dt &= \int_{-\infty}^{\infty} \left( \int_{-\infty}^{\infty} G_1(f)e^{j2\pi ft} \, df \right) \left( \int_{-\infty}^{\infty} G_2^*(f)e^{-j2\pi ft} \, df \right) \, dt \\
        &= \int_{-\infty}^{\infty} G_1(f)G_2^*(f) \left( \int_{-\infty}^{\infty} e^{j2\pi ft}e^{-j2\pi ft} \, dt \right) \, df \\
        &= \int_{-\infty}^{\infty} G_1(f)G_2^*(f) \left( \int_{-\infty}^{\infty} e^{(j2\pi ft - j2\pi ft)} \, dt \right) \, df \\
        &= \int_{-\infty}^{\infty} G_1(f)G_2^*(f) \left( \int_{-\infty}^{\infty} e^{0} \, dt \right) \, df \\
        &= \int_{-\infty}^{\infty} G_1(f)G_2^*(f) \left( \int_{-\infty}^{\infty} 1 \, dt \right) \, df \\
        &= \int_{-\infty}^{\infty} G_1(f)G_2^*(f) \, df
    \end{align*}
    And thus, we have shown that the left-hand side is equal to the right-hand side.

    \item \textit{Explain how we can obtain Parseval's Theorem from (1).}
    To find Parseval's Theorem, we can set $g_1(t) = g_2(t) = g(t)$, which implies that $G_1(f) = G_2(f) = G(f)$. Substituting these values into (1.1), we get:
    \begin{align*}
        \int_{-\infty}^{\infty} g(t)g^*(t) \, dt &= \int_{-\infty}^{\infty} G(f)G^*(f) \, df
    \end{align*}
    \item Using Parseval's Theorem, show that for any $k>0$ we have:
    \begin{align*}
        \int_{-\infty}^{\infty} \text{sinc}^2(kt) \, dt = \frac{1}{k}
    \end{align*}

    Here we have:
    \begin{align*}
        g(t) = \text{sinc}(kt), \quad G(f) = \frac{1}{k}\text{rect}\left(\frac{f}{k}\right)
    \end{align*}

    The conjugate of a rect function is itself, so we have:
    \begin{align*}
        \int_{-\infty}^{\infty} \text{sinc}^2(kt) \, dt &= \int_{-\infty}^{\infty} \frac{1}{k}\text{rect}\left(\frac{f}{k}\right)\frac{1}{k}\text{rect}\left(\frac{f}{k}\right) \, df \\
        &= \frac{1}{k^2}\int_{-k/2}^{k/2} \text{rect}\left(\frac{f}{k}\right) \, df \\
        &= \frac{1}{k^2}\int_{-k/2}^{k/2} 1 \, df \\
        &= \frac{1}{k^2}\left[ f \right]_{-k/2}^{k/2} \\
        &= \frac{1}{k^2}\left( \frac{k}{2} - \left( -\frac{k}{2} \right) \right) \\
        &= \frac{1}{k}
    \end{align*}

    And we have shown that $\int_{-\infty}^{\infty} \text{sinc}^2(kt) \, dt = \frac{1}{k}$.
\end{enumerate}

\section{Problem 2}
\begin{enumerate}[label=2.\arabic*]
    \item Show that:
    \begin{align*}
        r_{xy}(t) = \int_{-\infty}^{\infty} x(\tau)y^*(\tau - t) \, d\tau = \int_{-\infty}^{\infty} y^*(\tau)x(\tau + t) \, d\tau
    \end{align*}

    Since we know the correlation function is defined as:
    \begin{align*}
        r_{xy}(t) &= x(t) \ast y^*(-t) \\
        &= \int_{-\infty}^{\infty} x(\tau)y^*(\tau - t) \, d\tau, \quad \text{Let } u = \tau - t, \, du = d\tau \\ 
        &= \int_{-\infty}^{\infty} x(u + t)y^*(u) \, du \\
        &= \int_{-\infty}^{\infty} y^*(u)x(u + t) \, du, \quad \text{Let } \tau = u, \, d\tau = du \\
        &= \int_{-\infty}^{\infty} y^*(\tau)x(\tau + t) \, d\tau
    \end{align*}

    And we have shown as required.

    \item Show that $r_{xy}(t) = r_{yx}(-t)^\ast$:
    \begin{align*}
        r_{xy}(t) = \int_{-\infty}^{\infty} x(\tau)y^*(\tau - t) \, d\tau &= \int_{-\infty}^{\infty} y^*(\tau)x(\tau + t) \, d\tau \\
        &= \left(\int_{-\infty}^{\infty} y(\tau)x^*(\tau + t) \, d\tau\right)^\ast \\
        &= \left(r_{yx}(-t)\right)^\ast
    \end{align*}

    \item If $y(t) = x(t+T)$, we can express $r_{xy}(t)$ and $r_{yy}(t)$ in terms of $r_{xx}(t)$:
    
    First, $r_{xy}(t)$:
    \begin{align*}
        r_{xy}(t) &= x(t) \ast y^*(-t) = x(t) \ast x^*(T - t) \\
        &= \int_{-\infty}^{\infty} x(\tau)x^*(\tau - t + T) \, d\tau \\
        &= r_{xx}(t - T)
    \end{align*}

    Now, $r_{yy}(t)$:
    \begin{align*}
        r_{yy}(t) &= y(t) \ast y^*(-t) = x(t+T) \ast x^*(T - t) \\
        &= \int_{-\infty}^{\infty} x(\tau + T)x^*(\tau - t + T) \, d\tau, \quad \text {Let } \tau' = \tau + T, \, d\tau' = d\tau \\
        &= \int_{-\infty}^{\infty} x(\tau')x^*(\tau' - t) \, d\tau' \\
        &= r_{xx}(t)
    \end{align*}

    \item What is the relationship between the cross-ESD's $\Psi_{xy}(f)$ and $\Psi_{yx}(f)$?

    Given that $\Psi_{xy}(f) = \mathcal{F}\{r_{xy}(t)\}$ and $\Psi_{yx}(f) = \mathcal{F}\{r_{yx}(t)\}$. And from above we know that $r_{xy}(t) = r_{yx}(-t)^\ast$. We have:
    \begin{align*}
        \Psi_{xy}(f) = \mathcal{F}\{r_{xy}(t)\} &= \mathcal{F}\{r_{yx}(-t)^\ast\} \\
        &= \int_{-\infty}^{\infty} r_{yx}(-t)^\ast e^{-j2\pi ft} \, dt \\
        &= \int_{-\infty}^{\infty} \left(r_{yx}(-t) e^{j2\pi ft}\right)^\ast \, dt \\
        &= \left(\int_{-\infty}^{\infty} r_{yx}(-t) e^{j2\pi ft} \, dt\right)^\ast \\
        &= \left(\mathcal{F}\{r_{yx}(t)\}\right)^\ast \\
       \Psi_{xy}(f) &= \Psi_{yx}(f)^\ast
    \end{align*}

    \item We can find an expression of $\Psi_{xy}(f)$ in terms of $X(f)$ and $Y(f)$ as follows:

    \begin{align*}
        \Psi_{xy}(f) = \mathcal{F}\{r_{xy}(t)\} &= \mathcal{F}\{x(t) \ast y^*(-t)\} \\
        &= \mathcal{F}\{x(t)\} \cdot \mathcal{F}\{y^*(-t)\} \\
        &= X(f)Y^*(-f)
    \end{align*}

    \item We can show that the ESD is real and positive for every $f$ as follows:
    \begin{align*}
        \Psi_{xx}(f) = \mathcal{F}\{r_{xx}(t)\} &= \mathcal{F}\{x(t) \ast x^*(-t)\} \\
        &= \mathcal{F}\{x(t)\} \cdot \mathcal{F}\{x^*(-t)\} \\
        &= X(f)X^*(f) \\
        &= |X(f)|^2
    \end{align*}

    Since the magnitude squared of a complex number is always real and positive, we have shown that the ESD is real and positive for every $f$.

    \item We can find the expressions of $\Psi_{xy}(f)$ and $\Psi_{yy}(f)$ in terms of $\Psi_{xx}(f)$ and $H(f)$ as follows.

    First, $\Psi_{xy}(f)$. We know that $y(t) = h(t) \ast x(t)$, so we have:
    \begin{align}
        \Psi_{xy}(f) = \mathcal{F}\{r_{xy}(t)\} = \mathcal{F}\{x(t) \ast y^*(-t)\} = \mathcal{F}\{x(t) \ast h^*(-t) \ast x^*(-t)\}
    \end{align}
\end{enumerate}

% --------------------------------------------------------------------------------
% END BODY
% --------------------------------------------------------------------------------

\end{document}
ßf