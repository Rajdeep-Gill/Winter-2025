\documentclass{article}
\usepackage[margin=2.5cm, top=4cm, headheight=25pt]{geometry}
\usepackage{amsmath, amssymb, enumitem, fancyhdr, graphicx}
\usepackage[indent=20pt]{parskip}
\usepackage[hidelinks]{hyperref}
\usepackage{xcolor}
\usepackage{listings}
\usepackage{subcaption}
\usepackage{url}
\usepackage[most]{tcolorbox}
\usepackage{lastpage}
\usepackage{tikz}
\usetikzlibrary{arrows.meta}
\tcbuselibrary{listingsutf8} % Support for lstlistings within tcolorbox

\newtcolorbox[auto counter, number within=section]{question}[1][]{%
    colframe=gray!80,                      % Dark gray frame
    colback=gray!5,                       % Light gray background
    coltitle=black,                        % Black title
    title=\textbf{Question~\thetcbcounter}, % Bold title
    fonttitle=\bfseries\large,             % Subtle title font size
    rounded corners,                   % Slightly more rounded corners
    boxrule=0.25mm,                         % Thinner border for a sleek look
    enhanced,                              % Enhanced box features
    attach boxed title to top left={xshift=2mm, yshift=-2mm},
    boxed title style={colframe=gray!80, colback=gray!5, boxrule=0.25mm},
    % Title styling
    #1
}

\bibliographystyle{IEEEtran}
\graphicspath{{./images/}}

% -- Custom Variables --
\def\me{Rajdeep Gill 7934493}
\def\course{ECE 3760}
\def\labsection{A01}

\def\labno{5}
\def\title{Assignment 5: LTI Assignment}

% -- Styling for code snippets --
\lstset{
    basicstyle=\ttfamily\scriptsize,           % Basic font style
    keywordstyle=\color{blue},            % Keywords color
    commentstyle=\color{gray},            % Comments color
    stringstyle=\color{teal},             % Strings color
    numbers=left,                         % Line numbers on the left
    numberstyle=\tiny\color{gray},        % Line number style
    stepnumber=1,                         % Line number step
    numbersep=10pt,                       % Space between line numbers and code
    backgroundcolor=\color{lightgray!10}, % Background color
    frame=single,                         % Adds a frame around the code
    breaklines=true,                      % Line breaking for long lines
    captionpos=b,                         % Caption position
    showspaces=false,                     % Don't show spaces
    showstringspaces=false                % Don't show spaces in strings
}
\renewcommand{\lstlistingname}{Code Snippet}

\renewcommand{\arraystretch}{1.2} % For less-ugly tables
\setlength\parindent{0pt}

%----- Samples 
% Questions:
%   \begin{question}[title=Custom Question Title]
%       Question details
%   \end{question}

% Tables:
%   \begin{table}[htbp]
%       \centering
%       \caption{Table Caption}
%       \begin{tabular}{ll}
%           \toprule
%           \textbf{Column 1} & \textbf{Column 2} \\
%           \midrule
%           Row 1 & Row 2 \\
%           Row 3 & Row 4 \\
%           \bottomrule
%       \end{tabular}
%   \end{table} 

% Figures:
%   Single figure:
%       \begin{figure}[htbp]
%           \centering
%           \includegraphics[width=0.5\textwidth]{example-image}
%           \caption{Figure Caption}
%       \end{figure}
%   Multiple figures:
%       \begin{figure}[htbp]
%           \centering
%           \begin{subfigure}[b]{0.5\textwidth}
%               \includegraphics[width=\textwidth]{example-image-a}
%               \caption{First subfigure}
%           \end{subfigure}
%           \begin{subfigure}[b]{0.5\textwidth}
%               \includegraphics[width=\textwidth]{example-image-b}
%               \caption{Second subfigure}
%           \end{subfigure}
%           \caption{Main figure}
%       \end{figure}

\begin{document}

% --------------------------------------------------------------------------------
% TITLE
% --------------------------------------------------------------------------------

\begin{center}
    \huge \title

    \vspace{2mm}
    \hrule

    \vspace{4mm}
    \large \me

    \vspace{2mm}
    \large \course~\labsection

    \vspace{2mm}
    \today
\end{center}

\vspace{4mm}

% --------------------------------------------------------------------------------
% END TITLE
% --------------------------------------------------------------------------------

\newpage


\vspace{1cm}
\newpage

\pagestyle{fancy}
\fancyhead[L]{\large Assignment \labno}
\fancyhead[R]{\large \me}

\fancyfoot[C]{Page \thepage~of~\pageref{LastPage}}

% --------------------------------------------------------------------------------
% BODY
% --------------------------------------------------------------------------------
\textbf{Q1}

An LTI system is a Linear, Time-Invariant system. It follows the properites of linearity and time invariance. Linearity means the system is additive and homogeneous. Time invariance means the system's response to an input signal is independent of when the input signal is applied.

\textbf{Q2}

The system provided is a mapping as follows:

\[
    \begin{aligned}
        0 &\mapsto 5, \\
        1 &\mapsto 10, \\
        2 &\mapsto 15, \\
    \end{aligned}
\]
Equivalently as a function:
\begin{align*}
    y[n] = \begin{cases}
        5, & \text{if } n = 0, \\
        10, & \text{if } n = 1, \\
        15, & \text{if } n = 2, \\
    \end{cases}
\end{align*}

\begin{itemize}
    \item Is it deterministic? 
    
    Yes, the output is deterministic as it is a one-to-one mapping. So we can determine the output for any given input and vice-versa.

    \item Is it linear?
    
    The systen is not linear as it does not follow the properties of homogeneous and additive. For example if we give it an input scaled by $\alpha$, it is not the same as if we scaled the output by $\alpha$.

    Let the scaling factor be $\alpha = 2$:
    \begin{align*}
        y[1 \times \alpha] = y[2] = 15 \neq \alpha \times y[1] = 2 \times 10 = 20
    \end{align*}

    It also does not follow the additive property:
    \begin{align*}
        y[0 + 1] = y[1] = 10 \neq y[0] + y[1] = 5 + 10 = 15
    \end{align*}

    \item Is it time invariant?
    
    The system is time invariant as the output for a given input is independent of when the input is applied. 

    \item Is it causal?

    Yes, the system only depends on the current input value and not future values.

    \item Is it stable?

    Yes, the system is stable as the output values are bounded between $[0, 15]$ for the 3 input values.

    \item Is it discrete?

    Yes, the system is discrete as the input and output are discrete values. We only have the system defined for $n = 0, 1, 2$.

    \item Is it memoryless?

    Yes, the system is memoryless as the output is only dependent on the current input value.
\end{itemize}

% --------------------------------------------------------------------------------
% END BODY
% --------------------------------------------------------------------------------

\end{document}
