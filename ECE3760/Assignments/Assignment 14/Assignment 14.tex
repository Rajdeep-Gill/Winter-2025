\documentclass{article}
\usepackage[margin=2.5cm, top=4cm, headheight=25pt]{geometry}
\usepackage{amsmath, amssymb, enumitem, fancyhdr, graphicx}
\usepackage[indent=20pt]{parskip}
\usepackage[hidelinks]{hyperref}
\usepackage{xcolor}
\usepackage{listings}
\usepackage{subcaption}
\usepackage{url}
\usepackage[most]{tcolorbox}
\usepackage{lastpage}
\usepackage{tikz}
\usepackage{circuitikz}

\usetikzlibrary{arrows, positioning}
\tcbuselibrary{listingsutf8} % Support for lstlistings within tcolorbox

\newtcolorbox[auto counter, number within=section]{question}[1][]{%
    colframe=gray!80,                      % Dark gray frame
    colback=gray!5,                       % Light gray background
    coltitle=black,                        % Black title
    title=\textbf{Question~\thetcbcounter}, % Bold title
    fonttitle=\bfseries\large,             % Subtle title font size
    rounded corners,                   % Slightly more rounded corners
    boxrule=0.25mm,                         % Thinner border for a sleek look
    enhanced,                              % Enhanced box features
    attach boxed title to top left={xshift=2mm, yshift=-2mm},
    boxed title style={colframe=gray!80, colback=gray!5, boxrule=0.25mm},
    % Title styling
    #1
}

\bibliographystyle{IEEEtran}
\graphicspath{{./images/}}

% -- Custom Variables --
\def\me{Rajdeep Gill 7934493}
\def\course{ECE 3760}
\def\labsection{A01}

\def\labno{14}
\def\title{Assignment 14: Simplex and TMR Reliability}
% -- Styling for code snippets --
\lstset{
    basicstyle=\ttfamily\scriptsize,           % Basic font style
    keywordstyle=\color{blue},            % Keywords color
    commentstyle=\color{gray},            % Comments color
    stringstyle=\color{teal},             % Strings color
    numbers=left,                         % Line numbers on the left
    numberstyle=\tiny\color{gray},        % Line number style
    stepnumber=1,                         % Line number step
    numbersep=10pt,                       % Space between line numbers and code
    backgroundcolor=\color{lightgray!10}, % Background color
    frame=single,                         % Adds a frame around the code
    breaklines=true,                      % Line breaking for long lines
    captionpos=b,                         % Caption position
    showspaces=false,                     % Don't show spaces
    showstringspaces=false                % Don't show spaces in strings
}
\renewcommand{\lstlistingname}{Code Snippet}

\renewcommand{\arraystretch}{1.2} % For less-ugly tables
\setlength\parindent{0pt}

%----- Samples 
% Questions:
%   \begin{question}[title=Custom Question Title]
%       Question details
%   \end{question}

% Tables:
%   \begin{table}[htbp]
%       \centering
%       \caption{Table Caption}
%       \begin{tabular}{ll}
%           \toprule
%           \textbf{Column 1} & \textbf{Column 2} \\
%           \midrule
%           Row 1 & Row 2 \\
%           Row 3 & Row 4 \\
%           \bottomrule
%       \end{tabular}
%   \end{table} 

% Figures:
%   Single figure:
%       \begin{figure}[htbp]
%           \centering
%           \includegraphics[width=0.5\textwidth]{example-image}
%           \caption{Figure Caption}
%       \end{figure}
%   Multiple figures:
%       \begin{figure}[htbp]
%           \centering
%           \begin{subfigure}[b]{0.5\textwidth}
%               \includegraphics[width=\textwidth]{example-image-a}
%               \caption{First subfigure}
%           \end{subfigure}
%           \begin{subfigure}[b]{0.5\textwidth}
%               \includegraphics[width=\textwidth]{example-image-b}
%               \caption{Second subfigure}
%           \end{subfigure}
%           \caption{Main figure}
%       \end{figure}

\begin{document}

% --------------------------------------------------------------------------------
% TITLE
% --------------------------------------------------------------------------------

\begin{center}
    \huge \title

    \vspace{2mm}
    \hrule

    \vspace{4mm}
    \large \me

    \vspace{2mm}
    \large \course~\labsection

    \vspace{2mm}
    \today
\end{center}

\vspace{4mm}

% --------------------------------------------------------------------------------
% END TITLE
% --------------------------------------------------------------------------------

\newpage


\vspace{1cm}
\newpage

\pagestyle{fancy}
\fancyhead[L]{\large Assignment \labno}
\fancyhead[R]{\large \me}

\fancyfoot[C]{Page \thepage~of~\pageref{LastPage}}

% --------------------------------------------------------------------------------
% BODY
% --------------------------------------------------------------------------------
\section{Mean Time To Failure (MTTF)}
\subsection{MTTF Simulation}
We first simulate the time to failure of a simplex system. If we have a system with a constant failure rate of $\lambda = 0.01$, we can generate the time it takes for a module to fail. The following code gives us the time it takes for a module to fail:
\begin{lstlisting}[language=Python, caption={Simplex Simulation}, label={lst:python_code}]
p = 0.01 # Probability / lambda
num_iterations = 10000

def simulate_simplex(p):
    time_steps = 0
    while True:
        time_steps += 1
        if np.random.rand() < p:
            break
    return time_steps
\end{lstlisting}

We loop until the system fails at each iteration, and then take the average of the time steps. The average time as a result of the simulation was found to be:
\begin{verbatim}
Average time to failure (Simplex): 100.92 steps
\end{verbatim}

We can perform a similar simulation for a TMR system. The following code gives us the time it takes for a TMR system to fail:
\begin{lstlisting}[language=Python, caption={TMR Simulation}, label={lst:python_code}]
def simulate_tmr(p):
    time_steps = 0
    working_modules = 3
    while working_modules >= 2:
        time_steps += 1
        failures = np.random.rand(working_modules) < p
        working_modules -= np.sum(failures)
    return time_steps
\end{lstlisting}

At each time step, we find 3 random numbers and if any of them are less than the failure rate, we assume a module will fail. We then decrease the number of working modules by the number of failures and stop when we have less than 2 working modules. The average time as a result of the simulation was found to be:
\begin{verbatim}
Average time to failure (TMR): 83.38 steps
\end{verbatim}

Lastly, a system which starts out as a TMR system and switches to a simplex system when one of the modules fails can also be simulated in a similar manner. The following code gives us the time it takes for a TMR system to fail:
\begin{lstlisting}[language=Python, caption={Redundant-to-Simplex Simulation}, label={lst:python_code}]
def simulate_redundant_to_simplex(p):
    time_steps = 0
    active_modules = 3
    switched_to_simplex = False

    while True:
        time_steps += 1
        failures = np.random.rand(active_modules) < p
        active_modules -= np.sum(failures)

        if not switched_to_simplex and active_modules < 3:
            active_modules = 1  # Switch to simplex
            switched_to_simplex = True

        if active_modules <= 0:
            break

    return time_steps
\end{lstlisting}

We loop until one of the systems fails, and then we switch to a simplex system. This was done by setting the active modules to 1, and then just simulating the random generation on a single module at each time step. The average time as a result of the simulation was found to be:
\begin{verbatim}
Average time to failure (Redundant-to-Simplex): 133.45 steps
\end{verbatim}

\subsection{MTTF Calculations}
Given that we have a constant failure rate of $\lambda = 0.01$ failures per day, our reliability function is given by:
\begin{equation}
    R(t) = e^{-\lambda t} = e^{-0.01t}
\end{equation}

The mean time to failure (MTTF) is simply the integral of the reliability function over time, which can be expressed as:
\begin{align*}
    \text{MTTF} = \int_{0}^{\infty} R(t) dt &= \int_{0}^{\infty} e^{-0.01t} dt \\
    &= \left[-\frac{1}{0.01} e^{-0.01t}\right]_{0}^{\infty} \\
    &= 1/0.01 = 100 \text{ days}
\end{align*}
This matches the result of the simulation, where we found the average time to failure of a simplex system to be 100.92 steps.

Assuming a TMR system with perfect voting with 3 modules. The system is reliable if 2 of the 3 agree. The system will fail if more than 1 module failes, this gives us the following reliability function:
\begin{align*}
    R(t)_\text{single} = e^{-\lambda t} \\
\end{align*}
We can find the reliability by first finding the probability of 2 out of 3 modules working and the probability of all 3 modules working.
\begin{align*}
    P(\text{2 out of 3}) &= \binom{3}{2} R(t)^2 (1-R(t)) \\
    &= 3 R(t)^2 (1-R(t)) \\
    &= 3 (e^{-\lambda t})^2 (1-e^{-\lambda t}) \\
    &= 3e^{-2\lambda t} - 3e^{-3\lambda t} \\
\end{align*}

And when all 3 modules are working:
\begin{align*}
    P(\text{3 out of 3}) &= \binom{3}{3} R(t)^3 \\
    &= e^{-3\lambda t} 
\end{align*}

Therefore, the total reliability of the TMR system is given by:
\begin{align*}
    R(t) &= P(\text{2 out of 3}) + P(\text{3 out of 3}) \\
    &= 3e^{-2\lambda t} - 3e^{-3\lambda t} + e^{-3\lambda t} \\
    &= 3e^{-2\lambda t} - 2e^{-3\lambda t} \\
\end{align*}

The MTTF of the TMR system can be calculated as:
\begin{align*}
    \int_{0}^{\infty} R(t) dt &= \int_{0}^{\infty} (3e^{-2\lambda t} - 2e^{-3\lambda t}) dt \\
    &= \frac{3}{2\lambda} - \frac{2}{3\lambda} \\
    &= \frac{3}{2(0.01)} - \frac{2}{3(0.01)} \\
    &= 150 - 66.67 \\
    &= 83.33 \text{ days}
\end{align*}
This closely matches the result of the simulation, where we found the average time to failure of a TMR system to be 83.38 steps.


If we are to switch to a simplex system when one of the modules fails, we can find the MTTF of the system by first finding the MTTF of all 3 modules working, and adding the MTTF of the simplex system. The MTTF of all 3 modules working is given by:
\begin{align*}
    \int_{0}^{\infty} R(t)^3 dt &= \int_{0}^{\infty} e^{-3\lambda t} dt \\
    &= \left[-\frac{1}{3\lambda} e^{-3\lambda t}\right]_{0}^{\infty} \\
    &= \frac{1}{3(0.01)} = 33.33 \text{ days}
\end{align*}
Therefore, the MTTF of the switching system is given by:
\begin{align*}
    \text{MTTF} &= \int_{0}^{\infty} R(t)^3 dt + \int_{0}^{\infty} R(t) dt \\
    &= \int_{0}^{\infty} e^{-3\lambda t} dt + \int_{0}^{\infty} e^{-\lambda t} dt \\
    &= 33.33 + 100 = 133.33 \text{ days}
\end{align*}

This also matches the result of the simulation, where we found the average time to failure of a redundant-to-simplex system to be 133.45 steps.



% --------------------------------------------------------------------------------
% END BODY
% --------------------------------------------------------------------------------

\end{document}
