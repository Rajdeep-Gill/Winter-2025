\documentclass{article}
\usepackage[margin=2.5cm, top=4cm, headheight=25pt]{geometry}
\usepackage{amsmath, amssymb, enumitem, fancyhdr, graphicx}
\usepackage[indent=20pt]{parskip}
\usepackage[hidelinks]{hyperref}
\usepackage{xcolor}
\usepackage{listings}
\usepackage{subcaption}
\usepackage{url}
\usepackage[most]{tcolorbox}
\usepackage{lastpage}
\usepackage{tikz}
\usepackage{circuitikz}

\usetikzlibrary{arrows, positioning}
\tcbuselibrary{listingsutf8} % Support for lstlistings within tcolorbox

\newtcolorbox[auto counter, number within=section]{question}[1][]{%
    colframe=gray!80,                      % Dark gray frame
    colback=gray!5,                       % Light gray background
    coltitle=black,                        % Black title
    title=\textbf{Question~\thetcbcounter}, % Bold title
    fonttitle=\bfseries\large,             % Subtle title font size
    rounded corners,                   % Slightly more rounded corners
    boxrule=0.25mm,                         % Thinner border for a sleek look
    enhanced,                              % Enhanced box features
    attach boxed title to top left={xshift=2mm, yshift=-2mm},
    boxed title style={colframe=gray!80, colback=gray!5, boxrule=0.25mm},
    % Title styling
    #1
}

\bibliographystyle{IEEEtran}
\graphicspath{{./images/}}

% -- Custom Variables --
\def\me{Rajdeep Gill 7934493}
\def\course{ECE 3760}
\def\labsection{A01}

\def\labno{11}
\def\title{Assignment 11: Diffie-Hellman Key Exchange}
% -- Styling for code snippets --
\lstset{
    basicstyle=\ttfamily\scriptsize,           % Basic font style
    keywordstyle=\color{blue},            % Keywords color
    commentstyle=\color{gray},            % Comments color
    stringstyle=\color{teal},             % Strings color
    numbers=left,                         % Line numbers on the left
    numberstyle=\tiny\color{gray},        % Line number style
    stepnumber=1,                         % Line number step
    numbersep=10pt,                       % Space between line numbers and code
    backgroundcolor=\color{lightgray!10}, % Background color
    frame=single,                         % Adds a frame around the code
    breaklines=true,                      % Line breaking for long lines
    captionpos=b,                         % Caption position
    showspaces=false,                     % Don't show spaces
    showstringspaces=false                % Don't show spaces in strings
}
\renewcommand{\lstlistingname}{Code Snippet}

\renewcommand{\arraystretch}{1.2} % For less-ugly tables
\setlength\parindent{0pt}

%----- Samples 
% Questions:
%   \begin{question}[title=Custom Question Title]
%       Question details
%   \end{question}

% Tables:
%   \begin{table}[htbp]
%       \centering
%       \caption{Table Caption}
%       \begin{tabular}{ll}
%           \toprule
%           \textbf{Column 1} & \textbf{Column 2} \\
%           \midrule
%           Row 1 & Row 2 \\
%           Row 3 & Row 4 \\
%           \bottomrule
%       \end{tabular}
%   \end{table} 

% Figures:
%   Single figure:
%       \begin{figure}[htbp]
%           \centering
%           \includegraphics[width=0.5\textwidth]{example-image}
%           \caption{Figure Caption}
%       \end{figure}
%   Multiple figures:
%       \begin{figure}[htbp]
%           \centering
%           \begin{subfigure}[b]{0.5\textwidth}
%               \includegraphics[width=\textwidth]{example-image-a}
%               \caption{First subfigure}
%           \end{subfigure}
%           \begin{subfigure}[b]{0.5\textwidth}
%               \includegraphics[width=\textwidth]{example-image-b}
%               \caption{Second subfigure}
%           \end{subfigure}
%           \caption{Main figure}
%       \end{figure}

\begin{document}

% --------------------------------------------------------------------------------
% TITLE
% --------------------------------------------------------------------------------

\begin{center}
    \huge \title

    \vspace{2mm}
    \hrule

    \vspace{4mm}
    \large \me

    \vspace{2mm}
    \large \course~\labsection

    \vspace{2mm}
    \today
\end{center}

\vspace{4mm}

% --------------------------------------------------------------------------------
% END TITLE
% --------------------------------------------------------------------------------

\newpage


\vspace{1cm}
\newpage

\pagestyle{fancy}
\fancyhead[L]{\large Assignment \labno}
\fancyhead[R]{\large \me}

\fancyfoot[C]{Page \thepage~of~\pageref{LastPage}}

% --------------------------------------------------------------------------------
% BODY
% --------------------------------------------------------------------------------

\section{Key Exchange Algorithm}

We will pick a $q > 29$, so lets pick $q=41$. We first check if $q$ is prime, that is:
\begin{align*}
    2^{40} &\bmod 41 = 1 \\
    &\implies 2^{40} \equiv 1 \pmod{41} \\
    & \implies q \text{ is prime}
\end{align*}
We let $p = 2q + 1 = 83$. We then check if $p$ is prime:
\begin{align*}
    2^{82} &\bmod 83 = 1 \\
    &\implies 2^{82} \equiv 1 \pmod{83} \\
    &\implies p \text{ is prime}
\end{align*}

Now, we pick a generator $g$ that meets the following condition. We will start with $g = 3$:
\begin{align*}
    g^{(p-1)/2} &\not\equiv 1 \pmod{p} \\
    3^{(83-1)/2} &\bmod 83 = 41 \\
    &\implies 3^{41} \not\equiv 1 \pmod{83} \\
    &\implies g = 3
\end{align*}

So we can now share $g = 3$ and $p = 83$. Alice will now select a number $\textbf{a}$ from $2, 3, 4, \ldots, 81$ and Bob will select a number $\textbf{b}$ from $2, 3, 4, \ldots, 81$. Let Alice select $a = 5$ and Bob select $b = 7$ which they will keep secret. Then we calculate $A$ and $B$ to send to each other:
\begin{align*}
    A = g^a \bmod p = 3^5 \bmod 83 = 77 \\
    B = g^b \bmod p = 3^7 \bmod 83 = 29
\end{align*}


Alice and Bob both receive $A$ and $B$ from each other. They then calculate the shared secret key:
\begin{align*}
    \text{Alice: } B^a \bmod p = 29^5 \bmod 83 = 23 \\
    \text{Bob: } A^b \bmod p = 77^7 \bmod 83 = 23 \\
    \text{In Binary: } 0001 \, 0111
\end{align*}

Now suppose we want to share a message $M = 11 = 1011$b from Alice to Bob. Alice first XORs the message with the key and transmits the result:
\begin{align*}
    \text{Encrypted Message: } 0000 \, 1011 \oplus 0001 \, 0111 = 0001 \, 1100
\end{align*}

On the receiving end, Bob will XOR the received message with the shared key to decrypt the message:
\begin{align*}
    \text{Decrypted Message: } 0001 \, 1100 \oplus 0001 \, 0111 = 0000 \, 1011
\end{align*}

The Encrypted message is visible to anyone who intercepts it, but without the shared key, which is calculated using the private numbers $a$ and $b$, the message is secure.

% --------------------------------------------------------------------------------
% END BODY
% --------------------------------------------------------------------------------

\end{document}
