\documentclass{article}
\usepackage[margin=2.5cm, top=4cm, headheight=25pt]{geometry}
\usepackage{amsmath, amssymb, enumitem, fancyhdr, graphicx}
\usepackage[indent=20pt]{parskip}
\usepackage[hidelinks]{hyperref}
\usepackage{xcolor}
\usepackage{listings}
\usepackage{subcaption}
\usepackage{url}
\usepackage[most]{tcolorbox}
\usepackage{lastpage}
\usepackage{tikz}
\usepackage{circuitikz}

\usetikzlibrary{arrows, positioning}
\tcbuselibrary{listingsutf8} % Support for lstlistings within tcolorbox

\newtcolorbox[auto counter, number within=section]{question}[1][]{%
    colframe=gray!80,                      % Dark gray frame
    colback=gray!5,                       % Light gray background
    coltitle=black,                        % Black title
    title=\textbf{Question~\thetcbcounter}, % Bold title
    fonttitle=\bfseries\large,             % Subtle title font size
    rounded corners,                   % Slightly more rounded corners
    boxrule=0.25mm,                         % Thinner border for a sleek look
    enhanced,                              % Enhanced box features
    attach boxed title to top left={xshift=2mm, yshift=-2mm},
    boxed title style={colframe=gray!80, colback=gray!5, boxrule=0.25mm},
    % Title styling
    #1
}

\bibliographystyle{IEEEtran}
\graphicspath{{./images/}}

% -- Custom Variables --
\def\me{Rajdeep Gill 7934493}
\def\course{ECE 3760}
\def\labsection{A01}

\def\labno{}
\def\title{Physical Design Prototypes}
% -- Styling for code snippets --
\lstset{
    basicstyle=\ttfamily\scriptsize,           % Basic font style
    keywordstyle=\color{blue},            % Keywords color
    commentstyle=\color{gray},            % Comments color
    stringstyle=\color{teal},             % Strings color
    numbers=left,                         % Line numbers on the left
    numberstyle=\tiny\color{gray},        % Line number style
    stepnumber=1,                         % Line number step
    numbersep=10pt,                       % Space between line numbers and code
    backgroundcolor=\color{lightgray!10}, % Background color
    frame=single,                         % Adds a frame around the code
    breaklines=true,                      % Line breaking for long lines
    captionpos=b,                         % Caption position
    showspaces=false,                     % Don't show spaces
    showstringspaces=false                % Don't show spaces in strings
}
\renewcommand{\lstlistingname}{Code Snippet}

\renewcommand{\arraystretch}{1.2} % For less-ugly tables
\setlength\parindent{0pt}

%----- Samples 
% Questions:
%   \begin{question}[title=Custom Question Title]
%       Question details
%   \end{question}

% Tables:
%   \begin{table}[htbp]
%       \centering
%       \caption{Table Caption}
%       \begin{tabular}{ll}
%           \toprule
%           \textbf{Column 1} & \textbf{Column 2} \\
%           \midrule
%           Row 1 & Row 2 \\
%           Row 3 & Row 4 \\
%           \bottomrule
%       \end{tabular}
%   \end{table} 

% Figures:
%   Single figure:
%       \begin{figure}[htbp]
%           \centering
%           \includegraphics[width=0.5\textwidth]{example-image}
%           \caption{Figure Caption}
%       \end{figure}
%   Multiple figures:
%       \begin{figure}[htbp]
%           \centering
%           \begin{subfigure}[b]{0.5\textwidth}
%               \includegraphics[width=\textwidth]{example-image-a}
%               \caption{First subfigure}
%           \end{subfigure}
%           \begin{subfigure}[b]{0.5\textwidth}
%               \includegraphics[width=\textwidth]{example-image-b}
%               \caption{Second subfigure}
%           \end{subfigure}
%           \caption{Main figure}
%       \end{figure}

\begin{document}

% --------------------------------------------------------------------------------
% TITLE
% --------------------------------------------------------------------------------

\begin{center}
    \huge \title

    \vspace{2mm}
    \hrule

    \vspace{4mm}
    \large \me

    \vspace{2mm}
    \large \course~\labsection

    \vspace{2mm}
    \today
\end{center}

\vspace{4mm}

% --------------------------------------------------------------------------------
% END TITLE
% --------------------------------------------------------------------------------

\newpage


\vspace{1cm}
\newpage

\pagestyle{fancy}
\fancyhead[L]{\large \title}
\fancyhead[R]{\large \me}

\fancyfoot[C]{Page \thepage~of~\pageref{LastPage}}

% --------------------------------------------------------------------------------
% BODY
% --------------------------------------------------------------------------------
\section{Physical Design}
\subsection{Skip and Sweeper Devices}
The skip and sweeper devices share a similar overall design, with the key difference being the top section. The skip device, shown in \autoref{fig:skip_device}, has five larger holes to accommodate buttons for sending commands, while the sweeper device, shown in \autoref{fig:sweeper_device}, has twelve smaller holes arranged in a ring for the LED ring we used as the indicator.

Both devices have a modular design, with identical bottom halves that simplify assembly and reduce manufacturing costs. The top and bottom halves connect via matching screw threads, allowing for easy attachment and replacement if needed. Additionally, each device includes a ring for a lanyard that the skip can use to attach to a neck strap, or belt loop. For the sweeper this is an unecessary feature, but it is included for consistency in design. The sweeper's device would ideally be attached to the base of the broom handle so it is always in view.

\begin{figure}[ht!]
    \centering
    \begin{subfigure}{0.495\textwidth}
        \includegraphics[width=\textwidth]{skip_device.png}
        \caption{Skip Device}
        \label{fig:skip_device}
    \end{subfigure}
    \begin{subfigure}{0.495\textwidth}
        \includegraphics[width=\textwidth]{sweeper_device.png}
        \caption{Sweeper Device}
        \label{fig:sweeper_device}
    \end{subfigure}
    \caption{Skip and Sweeper Devices}
\end{figure}

\subsection{Mold Design}
The mold for the skip device's top part, shown in \autoref{fig:skip_top_mold}, is a two-part mold designed for easy removal of the finished piece. Alignment holes on both halves ensure proper positioning, while the pour hole, located on the top half, allows the material to fill from the inside out. This approach should help minimize visible imperfections on the exterior surface.

For the bottom part, shown in \autoref{fig:skip_bottom_mold}, is a three-part mold. The additional section accommodates the lanyard hook, ensuring it is molded as a single piece. As with the top mold, the material fills from the inside out to keep imperfections internal. While a two-part mold could be an alternative, it would require a different cutting geometry, whereas the current design maintains three flat parting surfaces. Similarly, the pour hole is positioned to fill the mold from the inside out, ensuring that any imperfections remain on the inside of the case.

\begin{figure}[ht!]
    \centering
    \begin{subfigure}{0.495\textwidth}
        \includegraphics[width=\textwidth]{skip_top_mold.png}
        \caption{Skip Top Mold}
        \label{fig:skip_top_mold}
    \end{subfigure}
    \begin{subfigure}{0.495\textwidth}
        \includegraphics[width=\textwidth]{skip_bottom_mold.png}
        \caption{Sweeper Top Mold}
        \label{fig:skip_bottom_mold}
    \end{subfigure}
\end{figure}



% --------------------------------------------------------------------------------
% END BODY
% --------------------------------------------------------------------------------

\end{document}
